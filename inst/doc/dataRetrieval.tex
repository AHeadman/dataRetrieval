%\VignetteIndexEntry{Introduction to the dataRetrieval package}
%\VignetteDepends{}
%\VignetteSuggests{}
%\VignetteImports{}
%\VignettePackage{}

\documentclass[a4paper,11pt]{article}

\usepackage{amsmath}
\usepackage{times}
\usepackage{hyperref}
\usepackage[numbers, round]{natbib}
\usepackage[american]{babel}
\usepackage{authblk}
\renewcommand\Affilfont{\itshape\small}
\usepackage{Sweave}
\renewcommand{\topfraction}{0.85}
\renewcommand{\textfraction}{0.1}
\usepackage{graphicx}


\textwidth=6.2in
\textheight=8.5in
\parskip=.3cm
\oddsidemargin=.1in
\evensidemargin=.1in
\headheight=-.3in

%------------------------------------------------------------
% newcommand
%------------------------------------------------------------
\newcommand{\scscst}{\scriptscriptstyle}
\newcommand{\scst}{\scriptstyle}
\newcommand{\Robject}[1]{{\texttt{#1}}}
\newcommand{\Rfunction}[1]{{\texttt{#1}}}
\newcommand{\Rclass}[1]{\textit{#1}}
\newcommand{\Rpackage}[1]{\textit{#1}}
\newcommand{\Rexpression}[1]{\texttt{#1}}
\newcommand{\Rmethod}[1]{{\texttt{#1}}}
\newcommand{\Rfunarg}[1]{{\texttt{#1}}}

\begin{document}
\Sconcordance{concordance:dataRetrieval.tex:dataRetrieval.Rnw:%
1 126 1 49 0 1 7 15 1 1 14 55 1 3 0 36 1 2 0 11 1 24 %
0 24 1 3 0 23 1 3 0 6 1 7 0 18 1 3 0 25 1 1 0 17 1 9 %
0 6 1 7 0 21 1 8 0 16 1 2 0 11 1 23 0 21 1 9 0 20 1 3 %
0 6 1 17 0 27 1 6 0 11 1 9 0 15 1 20 0 21 1 4 0 21 1 %
4 0 17 1 7 0 22 1 8 0 19 1 4 0 9 1 4 0 78 1 1 2 9 1 1 %
4 4 1 20 0 44 1 4 0 32 1 4 0 21 1 4 0 21 1 37 0 13 1 %
9 0 95 1 4 0 9 1 12 0 13 1 4 0 14 1 4 0 5 1 4 0 23 1 %
18 0 8 1 4 0 55 1}


%------------------------------------------------------------
\title{Introduction to the dataRetrieval package}
%------------------------------------------------------------
\author[1]{Laura De Cicco}
\author[1]{Robert Hirsch}
\affil[1]{United States Geological Survey}



\maketitle
\tableofcontents

%------------------------------------------------------------
\section{Introduction to dataRetrieval}
%------------------------------------------------------------ 
The dataRetrieval package was created to simplify the process of getting hydrologic data in the R enviornment. It has been specifically designed to work seamlessly with the EGRET package: Exploration and Graphics for RivEr Trends (EGRET) 

%------------------------------------------------------------
\subsection{What is dataRetrieval?}
%------------------------------------------------------------ 
The dataRetrieval package was created to simplify hydrologic data retrieval.  The options are web or user-produced files.....

\newpage
%------------------------------------------------------------ 
\section{Getting Started}
%------------------------------------------------------------ 
This section describes the options for downloading and installing the dataRetrieval package.
%------------------------------------------------------------
\subsection{Installing dataRetrieval from downloaded binary:}
%------------------------------------------------------------ 
The dataRetrieval pacakage is available for download at \url{https://github.com/USGS-CIDA/WRTDS/downloads}.  If the package's tar.gz file is saved in R's working directory, then the following commands will fully install the package:

\begin{Schunk}
\begin{Sinput}
> install.packages("dataRetrieval_1.2.0.tar.gz", 
+                  repos=NULL, type="source")
\end{Sinput}
\end{Schunk}

If the downloaded file is stored in an alternative location, include the path in the install command.  A Windows example looks like this (notice the direction of the slashes, they are in the opposite direction that Windows normally creates paths):

\begin{Schunk}
\begin{Sinput}
> install.packages(
+   "C:/RPackages/Statistics/dataRetrieval_1.2.0.tar.gz", 
+   repos=NULL, type="source")
\end{Sinput}
\end{Schunk}

A Mac example looks like this:

\begin{Schunk}
\begin{Sinput}
> install.packages(
+   "/Users/userA/RPackages/Statistic/dataRetrieval_1.2.0.tar.gz", 
+   repos=NULL, type="source")
\end{Sinput}
\end{Schunk}

Some users have found it necessary to delete the package folders before installing newer versions of either dataRetrieval or EGRET. If you are experiencing an issue after updating a package, trying deleting the package folder, the default location for Windows is something like this: C:/Users/ldecicco/Documents/R/win-library/2.15/dataRetrieval the default for a Mac: /Users/ldecicco/Library/R/2.15/library/dataRetrieval Then, re-install the package using the directions above. Moving to CRAN should solve this problem.


%------------------------------------------------------------
\subsection{Installing dataRetrieval from gitHub:}
%------------------------------------------------------------
Alternatively, R-developers can install the latest version of dataRetrieval directly from gitHub using the devtools package.  devtools is available on CRAN.  Simpley type the following commands into R to install the latest version of dataRetrieval available on gitHub.  Rtools (for Windows) and latex tools are required.

\begin{Schunk}
\begin{Sinput}
> library(devtools)
> install_github("dataRetrieval", "USGS-CIDA")
\end{Sinput}
\end{Schunk}
To then open the library, simpley type:

\begin{Schunk}
\begin{Sinput}
> library(dataRetrieval)
\end{Sinput}
\end{Schunk}

%------------------------------------------------------------
\subsection{A Simple Web Retrieval Example}
%------------------------------------------------------------ 
In this example, we use 3 dataRetrieval functions to get daily streamflow data and inorganic nitrogen sample results, and site information for a USGS gaging station with the ID 06934500.  The station is Missouri River at Hermann, MO (which is discovered in the INFO dataset).

\begin{Schunk}
\begin{Sinput}
> Daily <- getDVData("06934500","00060","1970-10-01","2011-09-30")
\end{Sinput}
\begin{Soutput}
There are  14975 data points, and  14975 days.
There are  0 zero flow days
If there are any zero discharge days, all days had 0 cubic meters per second added to the discharge value.
\end{Soutput}
\begin{Sinput}
> head(Daily)
\end{Sinput}
\begin{Soutput}
        Date        Q Julian Month Day  DecYear MonthSeq Qualifier i     LogQ
1 1970-10-01 3879.408  44102    10 274 1970.747     1450         A 1 8.263438
2 1970-10-02 3454.655  44103    10 275 1970.750     1450         A 2 8.147478
3 1970-10-03 3029.903  44104    10 276 1970.753     1450         A 3 8.016286
4 1970-10-04 2644.793  44105    10 277 1970.755     1450         A 4 7.880348
5 1970-10-05 2293.665  44106    10 278 1970.758     1450         A 5 7.737906
6 1970-10-06 2072.793  44107    10 279 1970.761     1450         A 6 7.636652
  Q7 Q30
1 NA  NA
2 NA  NA
3 NA  NA
4 NA  NA
5 NA  NA
6 NA  NA
\end{Soutput}
\begin{Sinput}
> Sample <-getSampleData("06934500","00631","1970-10-01","2011-09-30")
> head(Sample)
\end{Sinput}
\begin{Soutput}
        Date ConcLow ConcHigh Uncen ConcAve Julian Month Day  DecYear MonthSeq
1 1979-09-26    1.10     1.10     1    1.10  47384     9 269 1979.734     1557
2 1979-10-16    0.42     0.42     1    0.42  47404    10 289 1979.788     1558
3 1979-11-27    2.00     2.00     1    2.00  47446    11 331 1979.903     1559
4 1979-12-18    1.70     1.70     1    1.70  47467    12 352 1979.960     1560
5 1980-01-29    1.30     1.30     1    1.30  47509     1  29 1980.078     1561
6 1980-02-21    1.10     1.10     1    1.10  47532     2  52 1980.141     1562
       SinDY      CosDY
1 -0.9946999 -0.1028210
2 -0.9712570  0.2380333
3 -0.5724040  0.8199718
4 -0.2463613  0.9691781
5  0.4699767  0.8826788
6  0.7733507  0.6339785
\end{Soutput}
\begin{Sinput}
> INFO <-getMetaData("06934500","00631", interactive=FALSE)
> colnames(INFO)
\end{Sinput}
\begin{Soutput}
 [1] "agency.cd"             "site.no"               "station.nm"           
 [4] "site.tp.cd"            "lat.va"                "long.va"              
 [7] "dec.lat.va"            "dec.long.va"           "coord.meth.cd"        
[10] "coord.acy.cd"          "coord.datum.cd"        "dec.coord.datum.cd"   
[13] "district.cd"           "state.cd"              "county.cd"            
[16] "country.cd"            "map.nm"                "map.scale.fc"         
[19] "alt.va"                "alt.meth.cd"           "alt.acy.va"           
[22] "alt.datum.cd"          "huc.cd"                "basin.cd"             
[25] "topo.cd"               "construction.dt"       "inventory.dt"         
[28] "drain.area.va"         "contrib.drain.area.va" "tz.cd"                
[31] "local.time.fg"         "reliability.cd"        "project.no"           
[34] "queryTime"             "drainSqKm"             "staAbbrev"            
[37] "param.nm"              "param.units"           "paramShortName"       
[40] "paramNumber"           "constitAbbrev"        
\end{Soutput}
\begin{Sinput}
> INFO$station.nm
\end{Sinput}
\begin{Soutput}
[1] "Missouri River at Hermann, MO"
\end{Soutput}
\begin{Sinput}
> Sample <- mergeReport()
\end{Sinput}
\begin{Soutput}
 Discharge Record is 14975 days long, which is 41 years
 First day of the discharge record is 1970-10-01 and last day is 2011-09-30
 The water quality record has 437 samples
 The first sample is from 1979-09-26 and the last sample is from 2011-09-29
 Discharge: Minimum, mean and maximum 394 2660 20900
 Concentration: Minimum, mean and maximum 0.02 1.3 4.2
 Percentage of the sample values that are censored is 1.4 %
\end{Soutput}
\end{Schunk}


In the next section, we will go into detail the available functions in dataRetrieval, their required input and generated output.

\newpage
%------------------------------------------------------------
\section{Function Details}
%------------------------------------------------------------

%------------------------------------------------------------
\subsection{Daily Value Retrievals}
%------------------------------------------------------------


%------------------------------------------------------------
\subsection{Water Quality Retrievals}
%------------------------------------------------------------


%------------------------------------------------------------
\subsection{Site Information Retrievals}
%------------------------------------------------------------



\newpage

%------------------------------------------------------------
% BIBLIO
%------------------------------------------------------------
\begin{thebibliography}{10}

\bibitem{HirschI}
Helsel, D.R. and R. M. Hirsch, 2002. Statistical Methods in Water Resources Techniques of Water Resources Investigations, Book 4, chapter A3. U.S. Geological Survey. 522 pages. \url{http://pubs.usgs.gov/twri/twri4a3/}

\bibitem{HirschII}
Hirsch, R. M., Moyer, D. L. and Archfield, S. A. (2010), Weighted Regressions on Time, Discharge, and Season (WRTDS), with an Application to Chesapeake Bay River Inputs. JAWRA Journal of the American Water Resources Association, 46: 857-880. doi: 10.1111/j.1752-1688.2010.00482.x \url{http://onlinelibrary.wiley.com/doi/10.1111/j.1752-1688.2010.00482.x/full}

\bibitem{HirschIII}
Sprague, L. A., Hirsch, R. M., and Aulenbach, B. T. (2011), Nitrate in the Mississippi River and Its Tributaries, 1980 to 2008: Are We Making Progress? Environmental Science \& Technology, 45 (17): 7209-7216. doi: 10.1021/es201221s \url{http://pubs.acs.org/doi/abs/10.1021/es201221s}

\end{thebibliography}

\end{document}

\end{document}
